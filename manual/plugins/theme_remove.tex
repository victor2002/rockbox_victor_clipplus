% $Id: theme_remove.tex 29704 2011-04-11 20:58:19Z bertrik $ %
\subsection{\label{ref:ThemeRemove}Theme Remove}
This plugin offers a way to remove a theme. Open the \setting{Context Menu} (see \reference{ref:Contextmenu}) 
upon a theme\fname{.cfg} file and select \setting{Open With... $\rightarrow$ theme\_remove}.
Some files are not removed regardless of the \setting{Remove Options} such as
\fname{rockbox\_default.wps}\opt{lcd_bitmap}{ and the font file currently in use}.

\subsubsection{Theme Remove menu}
\begin{description}
  \item[Remove Theme.]
  Selecting this will delete the files specified in the \setting{Remove Options}.
  After a theme has been successfully removed, a log message is displayed listing 
  which items have been deleted and which are being kept. Exit this screen by 
  pressing any key. A file called \fname{theme\_remove\_log.txt} is created in 
  the root directory of your \dap{} listing all the changes.

  \item[Remove Options.]
  This menu specifies which items are removed if
  \setting{Remove Theme} is selected in the menu.

  One of the following options can be chosen for each setting.
  \begin{description}
    \item[Ask for Removal.]
    Selecting this option brings up a dialogue with two options:
    press \ActionYesNoAccept{} to confirm deletion or any other key to cancel.
    \item[Remove if not Used.]
    Selecting this option will remove the file automatically, if it is not 
    used by another theme in the theme directory and not currently used.
    \item[Never Remove.]
    Selecting this option will always skip deleting the file.
    \item[Always Remove.]
    Selecting this option will remove the file with no regard to
    whether it's used by another theme or not.
  \end{description}

  \begin{description}
\opt{lcd_bitmap}{
    \item[Font.]
    Specifies how the \fname{.fnt} file belonging to a theme \fname{.cfg} file is handled.
    If this option is set to \setting{Remove if not Used}, the fonts came from rockbox-fonts.zip
    will not be removed as themes may depend on those fonts.
}%
    \item[WPS.]
    Specifies how the \fname{.wps} file belonging to a theme \fname{.cfg} file is handled.
\opt{lcd_bitmap}{
    \item[Statusbar Skin.]
    Specifies how the \fname{.sbs} file belonging to a theme \fname{.cfg} file is handled.
}%
\opt{HAVE_REMOTE_LCD}{
    \item[Remote WPS.]
    Specifies how the \fname{.rwps} file belonging to a theme \fname{.cfg} file is handled.
    \item[Remote Statusbar Skin.]
    Specifies how the \fname{.rsbs} file belonging to a theme \fname{.cfg} file is handled.
}%
\opt{lcd_non-mono}{
    \item[Backdrop.]
    Specifies how the backdrop \fname{.bmp} file belonging to a theme \fname{.cfg} file is handled.
}%
\opt{lcd_bitmap}{
    \item[Iconset.]
    Specifies how the iconset \fname{.bmp} file belonging to a theme \fname{.cfg} file is handled.
    \item[Viewers Iconset.]
    Specifies how the viewers iconset \fname{.bmp} file belonging to a theme \fname{.cfg} file is handled.
}%
\opt{HAVE_REMOTE_LCD}{
    \item[Remote Iconset.]
    Specifies how the remote iconset \fname{.bmp} file belonging to 
    a theme \fname{.cfg} file is handled.
    \item[Remote Viewers Iconset.]
    Specifies how the remote viewers iconset \fname{.bmp} file belonging to 
    a theme \fname{.cfg} file is handled.
}% HAVE_REMOTE_LCD
\opt{lcd_color}{
    \item[Filetype Colours.]
     Specifies how the colours \fname{.colours} file belonging to a theme \fname{.cfg} file is handled.
}%

    \item[Create Log File.]
    Setting this to \setting{No} prevents the log file from being created.
  \end{description}
  \item[Quit.]
  Exits this plugin.
\end{description}
